\chapter{Description}

\postel allows extracting 2D vertical or horizontal cross sections
from the 3D result file. The resulting cross section files are readable with
the different post-processing tools of the \telemacsystem.

The files generated by \postel are in SERAFIN single precision format.

A computation is launched through the command \verb!postel3d.py [cas]! (cas:
the name of the steering file).

\postel is implemented like all the codes in the TELEMAC treatment chain.

Its execution is structured around the \telkey{STEERING FILE}, which is a
priori the single file which the user will have to consult and amend. It
gathers the names of all the files which define the computation to be carried
out.

The input files are:

\begin{itemize}
\item the \telkey{3D RESULT FILE} as provided by \telemac{3D}, in the \telemac{3D}
format,

\item the ASCII \telkey{FORTRAN FILE} containing the amended subroutines. That
file is optional.
\end{itemize}

The output files are:

\begin{itemize}
\item the \telkey{HORIZONTAL CROSS SECTION FILES}, in the SERAFIN format (can
be run under any post-processing tool that can read SERAFIN files),

\item the \telkey{VERTICAL CROSS SECTION FILES}, in the SERAFIN format (can be
run under any post-processing tool that can read SERAFIN files).
\end{itemize}


\chapter{The horizontal cross sections}

The horizontal cross sections are stored into binaries files in the SERAFIN
format, on the basis of one file per cross section. The name of each such file
consists in a common radical, which is given by the keyword \telkey{HORIZONTAL
CROSS SECTION FILE}, followed by an extension specifying the cross section
number.

A horizontal cross section is not necessarily horizontal, i.e. parallel to the
level Z~=~0. It may also be in the form of a 2D level of the \telemac{3D}
computation, with a possible vertical offset.

The mesh on which each of these cross sections is based is the 2D mesh which
was used for conducting the \telemac{3D} computation.

The horizontal cross sections are defined by means of the keywords:

\begin{itemize}
\item \telkey{NUMBER OF HORIZONTAL CROSS SECTIONS} specifies the number of
cross sections to be made,

\item \telkey{REFERENCE LEVEL FOR EACH HORIZONTAL CROSS SECTION} specifies the
\telemac{3D} vertical level from which the cross section shape is defined,

\item \telkey{ELEVATION FROM REFERENCE LEVEL} specifies the vertical distance
from the reference level at which the cross section has to be made.
\end{itemize}

The keyword \telkey{REFERENCE LEVEL} assumes a value ranging from 0 and NPLAN,
NPLAN is the number of levels selected for the \telemac{3D} computation which
made it possible to prepare the 3D result file read. If the value is between 1
and NPLAN, the reference level is the relevant mesh level which is liable to
move in time; if the value is zero, then the reference level is the level
Z~=~0. The cross section level is then inferred from the reference level
through a mere vertical translation the amount of which is set up by the
keyword \telkey{ELEVATION FROM REFERENCE LEVEL}.

In doing so, there is a slight difficulty because of the choice the user has to
define a cross section is through two parameters: \telkey{REFERENCE LEVEL FOR
EACH HORIZONTAL CROSS SECTION} and \telkey{ELEVATION FROM REFERENCE LEVEL}.
That only defines the cross section level, and in such a case, two problematic
scenarios can occur:

\begin{itemize}
\item either that level locally occurs above the free surface,
\item or it occurs locally under the bottom.
\end{itemize}

We have therefore adopted another approach which consists in performing a
linear extrapolation at these points from the closest 2 values, always
occurring vertically above that point. For those points located above the
surface, that extrapolation is then done from the values computed at the levels
NPLAN-1 and NPLAN whereas the points located below the bottom, the
extrapolation is computed from the values at levels 1 and 2.

Though the result of that extrapolation generally gives a value which is
realistic, since it is not far from the values found in the domain, the user
shall be made aware of the fact that such a result has no physical meaning and
that it does not occur on the result planes.

In order to indicate the locations of these nodes, an INDICATEUR\_DOM variable
is provided for the user in the cross section file. When that variable is
negative, that means the points are outside the domain. By inserting a coloured
(e.g. white) surface of the INDICATEUR\_DOM variable with a $]-\infty, 0]$
threshold, the user can then mask the out-of-domain areas that have meaningless
values.

As regards the VITESSE\_U and VITESSE\_V variables, we have preferred to preset
these variables to 0 at the points located outside the domain. That treatment
is more suitable for a vectorial plot.

\chapter{The vertical cross sections}

A vertical cross section can be defined in a 2D mesh as a sequence of linked
points making up a pecked line consisting of segments. That pecked line is
vertically extended from the surface down to the bottom. The minimum point
number is 2 (1 segment) and the maximum number is 9 (8 segments).

The vertical cross sections are defined by means of the keywords:

\begin{itemize}
\item \telkey{NUMBER OF VERTICAL CROSS SECTIONS} specifying the number of
cross sections to be made. That number cannot be in excess of 9. If over 9
vertical cross sections are desired, then several software executions should be
planned,

\item \telkey{NUMBER OF NODES FOR VERTICAL CROSS SECTION DISCRETIZATION}
setting the number of interpolation points in the horizontal direction of the
cross section. The points are evenly spaced. That number should be in excess of
2,

\item \telkey{ABSCISSAE OF THE VERTICES OF CROSS SECTION X} specifying the
abscissa of each point along the pecked line making up the vertical cross
section. That number should be higher than or equal to 2 and lower than or
equal to 9. \textit{X} specifies the cross section number ranging from 1 to 9.

\begin{WarningBlock}{Warning:}
If \textit{X} is higher than the \telkey{NUMBER OF VERTICAL CROSS
SECTIONS}, the information in that keyword will merely not be treated,
\end{WarningBlock}

\item \telkey{ORDINATES OF THE VERTICES OF CROSS SECTION X} specifying the
ordinate of each point along the pecked line making up the vertical cross
section. That number should be higher than or equal to 2 and lower than or
equal to 9. \textit{X} specifies the cross section number ranging from 1 to 9.

\begin{WarningBlock}{Warning:}
If \textit{X} is higher than the \telkey{NUMBER OF VERTICAL CROSS SECTIONS},
the information in that keyword will merely not be treated.
\end{WarningBlock}
\end{itemize}

The vertical cross sections are stored into binary files in the SERAFIN format,
on the basis of one file per cross section and per recorded time step. The name
of each such file consists of a common radical, as given by the keyword
\telkey{VERTICAL CROSS SECTION FILE}, followed by an extension specifying the
number of the cross section, then by an extension specifying the number of the
recorded time step. That increased number of files is necessary because the
meshes are distorted in time, and so are these cross sections, due to the free
surface motions.

The horizontal velocity components are given in a cross section-related
co-ordinate system provided for directly drawing the projection of the velocity
vector onto the cross section level. The components in the new co-ordinate
system are known as:

\begin{itemize}
\item VITESSE\_UT: tangential component,
\item VITESSE\_UN: normal component.
\end{itemize}

The \telemac{3D} computational domains often include a vertical scale which is
much lower than the horizontal scale. The vertical scale can be distorted by a
multiplicative factor so that RUBENS or any other post-processing tool, can
output a clearer display. That can be done using the keyword \telkey{DISTORTION
BETWEEN VERTICAL AND HORIZONTAL}.

\begin{WarningBlock}{Warning:}
If one uses a distortion factor, the vertical velocities will
themselves be multiplied by that factor. That is necessary to properly
represent the velocity vector directions.
\end{WarningBlock}

Lastly, note that the starting point is located to the left of the cross
section and the ending point to, the right of it i.e. the length is defined
from the bottom to the free surface.
