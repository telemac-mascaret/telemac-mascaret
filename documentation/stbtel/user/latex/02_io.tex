%--------------------------------------------------------------------------------
\chapter{Inputs and outputs}
%--------------------------------------------------------------------------------
%--------------------------------------------------------------------------------
\section{Preliminary remarks}
%--------------------------------------------------------------------------------
During a computation, the \stbtel software uses a number of input and output
files. Some of them are optional.\\
The input files are the following:
\begin{itemize}
\item The steering file,
\item The file containing the mesh to be treated (universal file),
\item One or many bottom topography files,
\item The Fortran file,
\item The optional data file.
The output files are the following:
\end{itemize}
\begin{itemize}
\item The geometry file according to the TELEMAC standard (Selafin)
\item The listing printout,
\item The boundary conditions file.
\end{itemize}

%--------------------------------------------------------------------------------
\section{The files}
%--------------------------------------------------------------------------------
%--------------------------------------------------------------------------------
\subsection{The steering file}
%--------------------------------------------------------------------------------
This is a text file created by the EDAMOX module or directly by the text
editor. In a way, it represents the control panel of the computation. It
contains a number of keywords to which values are assigned. If a keyword is not
contained in this file, \stbtel will assign it the default value defined in the
dictionary file (see description in section 2.2.9). If such a default value is
not defined in the dictionary file, the computation will stop with an error
message. For example, the command MESH = TRIGRID. enables the user to specify
that the computed file comes from the TRIGRID software.\\
\stbtel reads the steering file at the beginning of the computation.\\
The dictionary file and steering file are read by a utility called \damo,
which is included in \stbtel. Because of this, when the steering file is being
created, it is necessary to comply with the rules of syntax used in \damo
(this is done automatically if the file is created with EDAMOX). They are
briefly described below and an example is given in Appendix 4.\\

The rules of syntax are the following:
\begin{itemize}
\item The keywords may be of Integer, Real, Logical or Character type.
\item The order of keywords in the steering file is of no importance.
\item Each line is limited to 72 characters. However, it is possible to pass
from one line to the next as often as required, provided that the name of the
keyword is not split between two lines.
\item For keywords of the array type (only one-dimensional arrays are
available), the separator between two values is the semi-colon. It is not
necessary to give a number of values equal to the size of the array. In this
case, \damo returns the number of read values. For example:
\begin{lstlisting}[language=TelemacCas]
ABSCISSAE OF THE VERTICES OF THE POLYGON TO EXTRACT THE MESH  = 100.5; 500.6
\end{lstlisting}
  (This keyword is declared as an array of 9 values)
\item The signs ":" or "=" can be used indiscriminately as separator for the
name of a keyword and its value. They may be preceded or followed by any number
of spaces. The value itself may appear on the next line. For example:
\begin{lstlisting}[language=TelemacCas]
BOTTOM CORRECTION OF TRIGRID = 10.
\end{lstlisting}
or
\begin{lstlisting}[language=TelemacCas]
BOTTOM CORRECTION OF TRIGRID : 10.
\end{lstlisting}
or again
\begin{lstlisting}[language=TelemacCas]
BOTTOM CORRECTION OF TRIGRID  = 10.
\end{lstlisting}
\item Characters between two "/" on a line are considered as comments.
Similarly, characters between a "/" and the end of line are also considered as
comments. For example:
\begin{lstlisting}[language=TelemacCas]
BOTTOM CORRECTION OF TRIGRID = 280     / Bathy 1
\end{lstlisting}
\item A line beginning with "/" is considered to be all comment, even if there
is another "/" in the line. For example:
\begin{lstlisting}[language=TelemacCas]
      / The geometry file is ./mesh/geo
\end{lstlisting}
\item When writing integers, do not exceed the maximum size permitted by the
computer (for a computer with 32-bit architecture, the extreme values are -2
147 483 647 to + 2 147 483 648. Do not leave any space between the sign
(optional for the +) and number. A full stop is allowed at the end of a number.
\item When writing real numbers, the full stop and comma are accepted as
decimal points, as are E and D formats of FORTAN. ( 1.E-3  0.001  0,001  1.D-3
represent the same value).
\item When writing logical values, the following are acceptable: 1 OUI  YES
.TRUE.  TRUE  VRAI and 0 NON  NO  .FALSE.  FALSE  FAUX.
\end{itemize}
In addition to keywords, a number of instructions or meta-commands interpreted
during sequential reading of the steering file can also be used:
\begin{itemize}
\item Command \telkey{\&FIN} indicates the end of the file (even if the file is not
finished). This means that certain keywords can be deactivated simply by
placing them behind this command in order to reactivate them easily later on.
However, the computation continues.
\item Command \telkey{\&ETA} prints the list of keywords and the value that is assigned
to them when \damo encounters the command. This will be displayed at the
beginning of the listing printout (see section 2.2.7).
\item Command \telkey{\&LIS} prints the list of keywords. This will be displayed at the
beginning of the listing printout (see section 2.2.7).
\item Command \telkey{\&IND} prints a detailed list of keywords. This will be displayed
at the beginning of the listing printout (see section 2.2.7).
\item Command \telkey{\&STO} stops the program and the computation is interrupted.
\end{itemize}
The name of this file is given by using the keyword: \telkey{STEERING FILE}.

%--------------------------------------------------------------------------------
\subsection{The universal file}
%--------------------------------------------------------------------------------
This file holds the mesh to be treated. The software choosen for its creation
will change the format. Sometimes, it could be a file already fomated on the
\telemacsystem standard (Selafin) on which specials computations could be
treated.
In some case, this file could hold bathymetric informations.
The name of the file is given with the keyword: \telkey{UNIVERSAL FILE}

%--------------------------------------------------------------------------------
\subsection{The bottom topography files}
%--------------------------------------------------------------------------------
One of the goals of \stbtel is to interpolate a bathymetry point on the mesh
generator.  So, the bathymetric information must be given in one or in more
SINUSX format data files, or in X, Y Z files.
\stbtel can manage as many as 5 bottom topography files. In this case, it’s
important to take heed of the possible covering up of the zones in each file.
The names of this or these file(s) are given with the keyword: \telkey{BOTTOM
TOPOGRAPHY FILES}.
%--------------------------------------------------------------------------------
\subsection{The fortran user file}
%--------------------------------------------------------------------------------
As every TELEMAC simulation modules, \stbtel uses a Fortan user file.  With
\stbtel, It contains the main program. The role of this main program is simply
to determine the language used for writing the messages (English or French) and
to define the memory space by giving the size values for tables A (real) and IA
(integer).
If the size indicated by the user is too small, the \stbtel run is interrupted
and the software prints the minimum value to be put in the main program on the
listing printout. In the opposite case, the user recovers the exact size used
by the program, so that he can define the memory spaces as accurately as
possible, thus saving CPU memory.
This file is compiled and linked so as to generate the executable program for
the simulation.
The name of this file is given with the keyword: \telkey{FORTRAN FILE}.
An example of a FORTRAN file is given in appendix 5.
%--------------------------------------------------------------------------------
\subsection{The additional file of the mesh generator}
%--------------------------------------------------------------------------------
If there is an interface with the TRIGRID software, the file contains the
connectivity table necessary for the use of \stbtel.
If there is an interface with the FASTTBABS software, this file (which is
therefore optional) holds information about the type of the boundary
conditions.
The name of this file is given with the keyword: \telkey{MESH ADDITIONAL DATA
FILE}.
%--------------------------------------------------------------------------------
\subsection{The geometry file}
%--------------------------------------------------------------------------------
This geometry file is created by \stbtel from the universal file. This file is
done on the TELEMAC standard format (Selafin). It holds the information about
the geometry of the mesh, and possibly bathymetry information.
It’s a binary file that can be used with the RUBENS.
The name of this file is given with the keyword: \telkey{GEOMETRY FILE FOR TELEMAC}.

%--------------------------------------------------------------------------------
\subsection{The listing printout}
%--------------------------------------------------------------------------------
This is an \stbtel running report in which the user can find information about
operations performed by \stbtel.
The name of this file is managed directly by the \stbtel start-up procedure. In
general, it has the name of the steering file associated with the suffix
.sortie. A short example of a listing printout is given in appendix 6.
%--------------------------------------------------------------------------------
\subsection{The boundary conditions file}
%--------------------------------------------------------------------------------
This is a file generated by \stbtel. It can be modified using a text editor.
Each line of this file is dedicated at one point of the boundary. The numbering
of the boundary points is the same as the lines of the file.  It describes
first of all, the contour of the domain in a trigonometric direction from the
bottom left side point (minimum X +Y), then the islands on the clock wise.
For complete description of this file, see the \telemac{2} user manual.
If a mesh is being read with the Selafin format, the boundary conditions file
is not accurately completed by \stbtel (all the points are identified as walls
with slippery conditions).
The name of this file is given with the keyword: \telkey{BOUNDARY CONDITIONS
FILE}.
%--------------------------------------------------------------------------------
\subsection{The dictionary file}
%--------------------------------------------------------------------------------
This file contains all information on the keywords (name in French, name in
English, default values, type, documentation on keywords, information required
by EDAMOX). This file can be consulted by the user but must under no
circumstances be modified.
%--------------------------------------------------------------------------------
\subsection{The libraries}
%--------------------------------------------------------------------------------
When a computation is initiated, the main program written by the user is
compiled and then linked in order to generate the executable that is then run.
During the link edition phase, the following libraries are used:
\begin{itemize}
\item \stbtel: this library contains the subroutines that are specific to the
\stbtel code.
\item util: this library contains a number of utility subroutines, such as, for
example, the file read and writes routines.
\item damo: this library contains all subroutines that manage the reading of
keywords.
\item hp: this library contains the subroutines that manage the writing of the
various binaries (see section 2.3).
\end{itemize}
Usually, the user does not use other libraries than the standard ones. However,
the keyword LIBRARY is implemented to identify the used libraries on the
program generated. (see references manual for more details).
Moreover, the use of the last version installed on the computer is planned on
the software outline. The keyword RELEASE helps to identify the used library
version (for example to initiate a computation using a previous software
version). This keyword is described on the reference manual.
%--------------------------------------------------------------------------------
\section{File binaries}
%--------------------------------------------------------------------------------
The binary of a file is the method used by the computer for storing the
information physically on the disk (in contrast to storage in ASCII form, which
is used by the formatted files). \stbtel recognises three types of binary: the
standard binary of the computer on which the user is working, the IBM binary
enabling a file created on an IBM computer to be re-read, and the IEEE binary
that can be used, for example, to generate a file on a Cray OR IBM computer
that can be read by a workstation (provided that the appropriate subroutines
are included when \stbtel is installed on the computer).
The keyword used to fix the binary of the geometry file generated by \stbtel
is: \telkey{BINARY STANDARD}.
The default value specified on the dictionary file is “STD” (default binary of
the computer that is being used)

%--------------------------------------------------------------------------------
\section{Computer environment}
%--------------------------------------------------------------------------------
When using \stbtel on the main computer, the following keywords help to check
on the software computations. (These keywords are defined on the reference
manual).
\begin{itemize}
\item \telkey{KEYWORD}
\item \telkey{ACCOUNT}
\item \telkey{MEMORY SPACE}
\item \telkey{CPU TIME}
\item \telkey{USER}
\end{itemize}
