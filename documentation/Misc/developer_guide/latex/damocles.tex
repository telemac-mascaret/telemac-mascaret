%----------------------------------------------------------------------------------------------------
\chapter{Dictionary}
%----------------------------------------------------------------------------------------------------

This chapter will describe how the dictionary is used in \telemacsystem. The
dictionary is the input parameter for the reference documentation, the eficas
interface and the steering file. The module handling the dictionary is called
\damocles and can be found under \path{sources/utils/damocles}.

%----------------------------------------------------------------------------------------------------
\section{Description of the dictionary}
\label{ref:descDico}
%----------------------------------------------------------------------------------------------------
The dictionary is an ASCII file containing a list of keywords for each keyword
a number of parameters can be defined. Every line starting with a "/" is
considered a comment.\\
\begin{WarningBlock}{Warning}
No line of the dictionary should be longer that 72 characters.
\end{WarningBlock}

The first line of a dictionary must always be '\&DYN'

Here is a list of those parameters and theirs constraint. Every parameter
containing "(opt)" are optional the others are mandatory. The parameters must
be added in that order.

\begin{description}
\item[NOM] It is the name of the keyword in French. It should be max 72
  characters long and between "'". If the name contains an apostrophe it should
  be doubled.
\item[NOM1] It is the name of the keyword in English. It should be max 72
  characters long and between "'".
\item[TYPE] This is the type of the keyword the \telemacsystem handles 4 kind of types:
  \begin{description}
    \item[STRING] For text.
    \item[INTEGER] For integer values.
    \item[REAL] For real values.
    \item[LOGICAL] For boolean values.
  \end{description}
\item[INDEX] This is the index to access that keyword for each type it must be
  unique. Running \damocles can give you what indexes are available.
\item[TAILLE] Number of values expected for the keyword. If the keyword can
  have a dynamic number of values set TAILLE to 2 and add DYNLIST in APPARENCE.
\item[(opt) SUBMIT] This text is only for keywords referring to a file. It is a
  6 part chain separated by ";". Here is an example
  'T2DBI1-READ;T2DBI1;FACUL;BIN;LIT;SELAFIN'. Each part gives the following
  information:
  \begin{itemize}
    \item Name of the file in the temporary folder (6 letters long) - opening
      access (READ for read only, WRITE for write only, READWRITE for both).
    \item Name of the file in the temporary folder (6 letters long).
    \item Status of the file (FACUL if the file is optional, OBLIG if it is mandatory).
    \item Type of the file (BIN for binary, ASC for ASCII file).
    \item Opening access (LIT for read only, ECR, for write only, ECRLIT for both)
    \item Special treatment to apply for parallelism:
      \begin{itemize}
        \item SELAFIN-GEOM: geometry file will go through partel.
        \item SELAFIN: mesh file will go through but only the result field will
          be partitioned it will use the mesh from the SELAFIN-GEOM file.
        \item CONLIM: boundary condition file.
        \item PARAL: file will be copied for each processor.
        \item SCAL: no copy only the main processor will be able to access it.
        \item WEIRS, ZONES, SECTIONS: file added to partel with SELAFIN-GEOM.
        \item CAS: keyword for the steering file.
        \item DICO: keyword for the dictionary.
        \item FORTRAN: user fortran.
        \item DELWAQHYD, DELWAQSEG ...: files for coupling with Delwaq.
      \end{itemize}
  \end{itemize}
\item[DEFAUT] French default value of the keyword. It should contains TAILLE
  values.
\item[DEFAUT1] English default value of the keyword. It should contains TAILLE
  values.
\item[MNEMO] Name of the variable in the code containing the value of the keyword.
\item[(opt) CONTROLE] Two integer that define the min and max of the value of
  the keyword (This is not used yet).
\item[(opt) CHOIX] List of values available for the keyword (in French). If the
  keyword is a STRING just put the list of values in the following from
  'text1';'text2';...  You can also use CHOIX to build a association between
  the actual value and a string by following this syntax:
  'val1:"text1"';'val2:"text2"';... You can have a look at the keywords
  \telkey{SOLVER} and \telkey{VARIABLES FOR GRAPHIC PRINTOUTS} for examples.
  Note that the limitation to those choices is not made by \damocles when you
  are running a case, it is only done by the interface. Text for the value are
  not allowed to contain apostrophes.
\item[(opt) CHOIX1] Same as CHOIX but in English.
\item[(opt) APPARENCE] This is used to give information for the interface on
  how to enter the values for the keyword the possible values are:
  \begin{itemize}
    \item LIST: To display it in the form of an array.
    \item TUPLE: To display a couple of values.
    \item DYNLIST: To display a dynamic list.
  \end{itemize}
\item[RUBRIQUE] This is used to categorized the keywords. It should contain
  three values that are the name of the categories in French.\\
  \begin{WarningBlock}{Warning}
    The RUBRIQUE cannot have the same name as a keyword.
  \end{WarningBlock}
\item[RUBRIQUE1] Same RUBRIQUE but in English.
\item[(opt) COMPOSE] Not used yet.
\item[(opt) COMPORT] Not used yet.
\item[NIVEAU] This is the status of the keyword if 0 the keyword will be
  considered mandatory in the interface. Later it will also be used to group
  keywords for specific studies.
\item[AIDE] Text in French describing what the keyword is used for. This text
  can use LaTeX format. It will be used to generate the reference
  documentation.
\item[AIDE1] Same as AIDE but in English.
\end{description}

%----------------------------------------------------------------------------------------------------
\section{Description of the dictionary additional file}
\label{ref:descDicoAdd}
%----------------------------------------------------------------------------------------------------
This file is only for eficas it is not read during a basic run of
\telemacsystem. The file is build in two parts:
\begin{itemize}
  \item The dependencies between keywords (i.e keywords that should appear only
    if a keyword has a certain value) and "consigne" text that is to be
    displayed if the keyword as a certain value
  \item The status of RUBRIQUES.
\end{itemize}

The first part is a list of blocks for each keyword as they are linked to a keyword.
The block for a dependence must follow that syntax:
\begin{verbatim}
n number_of_the_dependence
condition
keyword_on_which_is_the_condition
keyword_1
keyword_2
...
keyword_n
\end{verbatim}
The \verb!number_of_dependence! will be increased as we add blocks for that
keyword.
The condition must follow a couple rules:
\begin{itemize}
  \item It must be in Python syntax.
  \item It must use the "eficas form of the keyword" (i.e. Replace " " "'" "-"
    by "\_").
  \item If the keyword has a CHOIX it must use the text and not the value.
\end{itemize}

If the condition is on more than one keyword and affect only one keyword set
\verb!number_of_the_dependencie! to 0 and \verb!n! to 1.

The block for a "consigne" must follow that syntax:
\begin{verbatim}
1 number_of_the_dependence
condition
keyword_on_which_is_the_condition
Text_in_French
Text_in_English
\end{verbatim}
The \verb!number_of_dependence! must be negative but still follow the increase
(i.e. if its the third dependence it will be -3). Both the text in French and
the one in English must be written on one line.

Here is an example for the keyword \telkey{INITIAL CONDITIONS} which has 3
dependencies and one "consigne".
\begin{verbatim}
2 1
INITIAL_CONDITIONS == 'CONSTANT ELEVATION'
INITIAL CONDITIONS
INITIAL ELEVATION
2 2
INITIAL_CONDITIONS == 'CONSTANT DEPTH'
INITIAL CONDITIONS
INITIAL DEPTH
2 3
INITIAL_CONDITIONS == 'TPXO SATELLITE ALTIMETRY'
INITIAL CONDITIONS
ASCII DATABASE FOR TIDE
1 -4
INITIAL_CONDITIONS == 'SPECIAL'
INITIAL CONDITIONS
Les conditions initiales sur la hauteur d''eau doivent etre precisees dans le sous-programme CONDIN.
The initial conditions with the water depth should be stated in the CONDIN subroutine.
\end{verbatim}

The second part that is at the end of the file begins with the line:
\begin{verbatim}
666 666
\end{verbatim}

Then it is a list of two lines:
\begin{verbatim}
NAME_OF_RUBRIQUE
STATUS
\end{verbatim}

By default all RUBRIQUE are mandatory here you can change that by setting
\verb!STATUS! to \verb!f!. \verb!NAME_OF_RUBRIQUE! must have the same name as
in the dictionary.

%----------------------------------------------------------------------------------------------------
\section{What \damocles can do for you}
%----------------------------------------------------------------------------------------------------

\damocles can be used via the script \verb!damocles.py! and has option for
three things:
\begin{itemize}
\item --dump, Read a dictionary and dump it back reordered and reorganize.
\item --eficas, Generate the Catalogue for eficas to build the interface.
\item --latex, Generate a LaTeX file for the reference documentation.
\end{itemize}
All those options are to be combine with the option \verb!-m! to specify on
which module to run the script by default it is run for all of them except
\mascaret.
The first one can also be used to get information on the dictionary such as the
index used, the RUBRIQUE in French and English to check that we have the same
number of each the reordered dictionary will be next to the original with a 2
added to the name. It also checks that the dictionary is following the rules
described before.

The eficas option is generating all the files needed to run the GUI. All the
files are added in the folder eficas in the source folder of the module.

The latex option is used by \verb!docTELEMAC.py! to compile the reference
manual for that module.
