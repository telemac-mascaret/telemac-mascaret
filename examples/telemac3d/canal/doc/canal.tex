% case name
\chapter{canal}
%
% - Purpose & Description:
%     These first two parts give reader short details about the test case,
%     the physical phenomena involved, the geometry and specify how the numerical solution will be validated
%
\section{Purpose}
%
This study case verifies that \telemac{3d} is able to compute the free
surface evolution along a channel with bottom friction.
%
\section{Description}
%
The chosen configuration is a straight channel 500~m long and 100~m wide
with a flat horizontal bottom (see figure 3.1.1).
Three different cases are studied:
\begin{itemize}
\item A 2D computation using \telemac{2d},
\item A 3D computation with hydrostatic option,
\item A 3D computation with non-hydrostatic option.
\end{itemize}
In all cases, the flow establishes a steady flow where the free surface
is influenced by the friction on the bottom.
%
% - Reference:
%     This part gives the reference solution we are comparing to and
%     explicits the analytical solution when available;
%
% bibliography can be here or at the end
%\subsection{Reference}
%
%
\subsection{Reference}
%

%
% - Geometry and Mesh:
%     This part describes the mesh used in the computation
%
%
\subsection{Geometry and Mesh}
%
\subsubsection{Bathymetry}
%
Flat horizontal bottom
%
\subsubsection{Geometry}
%
Channel length = 500~m\\
Channel width = 100~m
%
\subsubsection{Mesh}
%
551 triangular elements\\
319 nodes\\
10 levels regularly spaced on the vertical.
The repartition of the levels is presented on figure 3.1.1
%
% - Physical parameters:
%     This part specifies the physical parameters
%
%
\subsection{Physical parameters}
%
Horizontal constant viscosity:
\begin{itemize}
\item no with the hydrostatic option,
\item 0.1~m$^2$/s with the non-hydrostatic option
\end{itemize}
Vertical turbulence model: Nezu and Nakagawa mixing length model\\
Wind: no
%
% Experimental results (if needed)
%\subsection{Experimental results}
%
% bibliography can be here or at the end
%\subsection{Reference}
%
% Section for computational options
%\section{Computational options}
%
% - Initial and boundary conditions:
%     This part details both initial and boundary conditions used to simulate the case
%
%
\subsection{Initial and Boundary Conditions}
%
\subsubsection{Initial conditions}
%
Steady flow (3D cases initialised from the 2D case result file)
%
\subsubsection{Boundary conditions}
%
Upstream prescribed flow rate: 50~m$^3$/s\\
Downstream prescribed elevation: 0.5~m
%
\subsection{General parameters}
%
Time step: 2~s\\
Simulation duration: 4,000~s for the 2D computation then 2,000~s more
for 3D computations.
%
% - Numerical parameters:
%     This part is used to specify the numerical parameters used
%     (adaptive time step, mass-lumping when necessary...)
%
%
\subsection{Numerical parameters}
%
Advection of velocity: Characteristics
%
\subsection{Comments}
%
% - Results:
%     We comment in this part the numerical results against the reference ones,
%     giving understanding keys and making assumptions when necessary.
%
%
\section{Results}
%
On figure 3.1.2, the 3 free surfaces profiles corresponding to each
simulation are compared.
These three results are in agreement.
We can also observe that the flow is completely symmetrical without
any influence of the space discretisation.
%
\section{Conclusion}
%
\telemac{3d} is able to take into account correctly the bottom friction
term.
%
% Here is an example of how to include the graph generated by validateTELEMAC.py
% They should be in test_case/img
%\begin{figure} [!h]
%\centering
%\includegraphics[scale=0.3]{../img/mygraph.png}
% \caption{mycaption}\label{mylabel}
%\end{figure}
%
% bibliography
%\section{Reference}
