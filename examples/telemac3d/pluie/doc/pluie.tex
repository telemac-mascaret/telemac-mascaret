% case name
\chapter{pluie}
%
% - Purpose & Description:
%     These first two parts give reader short details about the test case,
%     the physical phenomena involved, the geometry and specify how the numerical solution will be validated
%
\section{Purpose}
%
This test illustrates that \telemac{3d} is able simulating a rain fall
(addition of fresh water on the surface).
%
\section{Description}
%
We consider a square basin with a water depth of 10~m.
The domain is initially at rest with a constant initial salinity equal
to 32~g/L.\\
A fictive rain of 864,000~mm per day (10~mm/s) is simulated during 3 s.
%
% - Reference:
%     This part gives the reference solution we are comparing to and
%     explicits the analytical solution when available;
%
% bibliography can be here or at the end
%\subsection{Reference}
%
%
\subsection{Reference}
%

%
% - Geometry and Mesh:
%     This part describes the mesh used in the computation
%
%
\subsection{Geometry and Mesh}
%
\subsubsection{Bathymetry}
%
Flat bottom
%
\subsubsection{Geometry}
%
Size of the basin domain = 10 m x 10 m (see figure 3.9.1)
%
\subsubsection{Mesh}
%
272 triangular elements\\
159 nodes\\
21 layers regularly spaced on the vertical
%
% - Physical parameters:
%     This part specifies the physical parameters
%
%
\subsection{Physical parameters}
%
Constant horizontal viscosity equal to 0.1~m$^2$/s\\
Mixing length on the vertical (Nezu-Nakagawa)\\
Coriolis: no\\
Wind: no\\
Rain: 864,000~mm/day (i.e. 10~mm/s)
%
% Experimental results (if needed)
%\subsection{Experimental results}
%
% bibliography can be here or at the end
%\subsection{Reference}
%
% Section for computational options
%\section{Computational options}
%
% - Initial and boundary conditions:
%     This part details both initial and boundary conditions used to simulate the case
%
%
\subsection{Initial and Boundary Conditions}
%
\subsubsection{Initial conditions}
%
No velocity\\
Constant water depth equal to 10~m\\
Salinity equal to 32~g/L
%
\subsubsection{Boundary conditions}
%
Solid boundary everywhere
%
\subsection{General parameters}
%
Time step: 1~s\\
Simulation duration: 3~s
%
% - Numerical parameters:
%     This part is used to specify the numerical parameters used
%     (adaptive time step, mass-lumping when necessary...)
%
%
\subsection{Numerical parameters}
%
Hydrostatic simulation\\
Advection of velocities: Characteristics\\
Advection of tracers: PSI-type MURD scheme
%
\subsection{Comments}
%
% - Results:
%     We comment in this part the numerical results against the reference ones,
%     giving understanding keys and making assumptions when necessary.
%
%
\section{Results}
%
The water mass balance is perfect (of the order of 10$^{-12}$).
The initial amount of water is 1,000~m$^3$.
The quantity of rain during the total simulation is 30~mm.
Taking into account the surface of the basin (100~m$^2$), the quantity
if supplied fresh water is equal to 3~m$^3$.
The log file of the simulation indicates a final mass equal to
1,003~m$^3$.
In addition, the total tracer (salinity) remains constant (of the order
of 10$^{-8}$).\\
Due to the rain, the salinity at the surface decreases during the
simulation (see figure 3.9.2).
The salinity profile on the vertical at the end of the simulation is
presented on figure 3.9.2.
%
\section{Conclusion}
%
\telemac{3d} is capable of simulating the supply of water on the free
surface due to the rain.
%
% Here is an example of how to include the graph generated by validateTELEMAC.py
% They should be in test_case/img
%\begin{figure} [!h]
%\centering
%\includegraphics[scale=0.3]{../img/mygraph.png}
% \caption{mycaption}\label{mylabel}
%\end{figure}
%
% bibliography
%\section{Reference}
